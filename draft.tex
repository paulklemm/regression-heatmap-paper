\documentclass[journal]{style/vgtc} 			          % final (journal style)
%\documentclass[review,journal]{style/vgtc}         % review (journal style)
%\documentclass[widereview]{style/vgtc}             % wide-spaced review
%\documentclass[preprint,journal]{style/vgtc}       % preprint (journal style)
%\documentclass[electronic,journal]{style/vgtc}     % electronic version, journal
%% Uncomment one of the lines above depending on where your paper is
%% in the conference process. ``review'' and ``widereview'' are for review
%% submission, ``preprint'' is for pre-publication, and the final version
%% doesn't use a specific qualifier. Further, ``electronic'' includes
%% hyperreferences for more convenient online viewing.
%% Please use one of the ``review'' options in combination with the
%% assigned online id (see below) ONLY if your paper uses a double blind
%% review process. Some conferences, like IEEE Vis and InfoVis, have NOT
%% in the past.

%% Please note that the use of figures other than the optional teaser is not permitted on the first page
%% of the journal version.  Figures should begin on the second page and be
%% in CMYK or Grey scale format, otherwise, colour shifting may occur
%% during the printing process.  Papers submitted with figures other than the optional teaser on the
%% first page will be refused.

%% These three lines bring in essential packages: ``mathptmx'' for Type 1
%% typefaces, ``graphicx'' for inclusion of EPS figures. and ``times''
%% for proper handling of the times font family.

\usepackage{mathptmx}
\usepackage{graphicx}
\usepackage{times}

% -- My Own Packages and Commands
\usepackage[normalem]{ulem}
\usepackage{xcolor}
\newcommand{\rem}[1]{\textcolor{red}{\sout{#1}}}
\newcommand{\add}[1]{\textcolor{blue}{\uline{#1}}}
\newcommand{\com}[1]{\textcolor{orange}{\uline{#1}}}

%% We encourage the use of mathptmx for consistent usage of times font
%% throughout the proceedings. However, if you encounter conflicts
%% with other math-related packages, you may want to disable it.

%% This turns references into clickable hyperlinks.
\usepackage[bookmarks,backref=true,linkcolor=black]{hyperref} %,colorlinks
\hypersetup{
  pdfauthor = {},
  pdftitle = {},
  pdfsubject = {},
  pdfkeywords = {},
  colorlinks=true,
  linkcolor= black,
  citecolor= black,
  pageanchor=true,
  urlcolor = black,
  plainpages = false,
  linktocpage
}

%% If you are submitting a paper to a conference for review with a double
%% blind reviewing process, please replace the value ``0'' below with your
%% OnlineID. Otherwise, you may safely leave it at ``0''.
%% TODO: Blind Review: Replace with online ID
\onlineid{175}

%% declare the category of your paper, only shown in review mode
\vgtccategory{Research}

%% allow for this line if you want the electronic option to work properly
\vgtcinsertpkg

%% In preprint mode you may define your own headline.
%\preprinttext{To appear in an IEEE VGTC sponsored conference.}

%% Paper title.

\title{Regression Cube Analysis of Cohort Study Data}

%% This is how authors are specified in the journal style

%% indicate IEEE Member or Student Member in form indicated below
\author{Paul Klemm, Kai Lawonn, Sylvia Gla{\ss}er, Uli Niemann, Katrin Hegenscheid, Henry V{\"o}lzke, Bernard Preim}
\authorfooter{
%% insert punctuation at end of each item
\item
 Paul Klemm, Kai Lawonn, Sylvia Gla{\ss}er, Uli Niemann, Bernhard Preim are with Otto-von-Guericke University Magdeburg, Germany. E-mail: \{klemm,lawonn,niemann,preim\}@ovgu.de
\item
 Katrin Hegenscheid, Henry V{\"o}lzke are with Ernst-Moritz-Arndt University Greifswald, Germany. E-mail: \{katrin.hegenscheid,voelzke\}@uni-greifswald.de
}

%other entries to be set up for journal
\shortauthortitle{Klemm \MakeLowercase{\textit{et al.}}: Regression Cube Analysis of Cohort Study Data}
%\shortauthortitle{Firstauthor \MakeLowercase{\textit{et al.}}: Paper Title}

%% Abstract section.
\abstract{%
\com{Problem}.
Epidemiological studies comprise heterogeneous data about a subject group (a \emph{cohort}) to define disease-specific risk factors.
%%
These data contain information (\emph{features}) about a subjects' lifestyle, medical conditions and also medical image data.
%%
These features are analyzed using statistical regression analysis to identify feature combinations indicating a disease (the \emph{target feature}).
%%
Although there is a strong demand for overview visualizations of a whole data set towards a target feature, no suitable tool is available for epidemiological researchers.

\com{New Solution}.
%%
We propose an analysis approach of epidemiological data sets by incorporating all features in an exhaustive regression-based analysis.
%%
This approach combines all \emph{independent features} with respect to a \emph{target feature} and provides a visualization, which reveals insight into the data by highlighting relationships.
%%
A 3D-visualization of all combinations of two to three independent features towards a target acts as an overview of the whole data set, the \emph{Regression Cube}.
%%
Slicing through the \emph{Regression Cube} allows for the detailed analysis of features towards the target disease.
%%
Expert knowledge about disease-specific hypotheses can be included into the analysis by adjusting the regression model formulas.
%%
Furthermore, the influences of features can be assessed using a difference view comparing different calculation results.
%%
\com{Validation \& Results}.
%%
We applied our \emph{Regression Cube} method to a hepatic steatosis data set to reproduce results from a data-mining driven analysis.
%%
A qualitative analysis with three domain experts was conducted on a breast fat data set.
%%
We were able to derive new hypotheses about relations between breast fat density and breast lesions towards breast cancer.
%%
\com{Implications}.
%%
With our work, we present a visual overview of epidemiological data with the \emph{Regression Cube}, which allows for the first time a interactive regression-based analysis of large feature sets with respect to a disease.
} % end of abstract

%% Keywords that describe your work. Will show as 'Index Terms' in journal
%% please capitalize first letter and insert punctuation after last keyword
\keywords{Interactive Visual Analysis, Epidemiology, Breast Cancer, Hepatic Steatosis}

%% ACM Computing Classification System (CCS). 
%% See <http://www.acm.org/class/1998/> for details.
%% The ``\CCScat'' command takes four arguments.

\CCScatlist{ % not used in journal version
 \CCScat{K.6.1}{Management of Computing and Information Systems}%
{Project and People Management}{Life Cycle};
 \CCScat{K.7.m}{The Computing Profession}{Miscellaneous}{Ethics}
}

%% Uncomment below to include a teaser figure.
  % \teaser{
  % \centering
  % \includegraphics[width=16cm]{CypressView}
  % \caption{In the Clouds: Vancouver from Cypress Mountain.}
  % }

%% Uncomment below to disable the manuscript note
%\renewcommand{\manuscriptnotetxt}{}

%% Copyright space is enabled by default as required by guidelines.
%% It is disabled by the 'review' option or via the following command:
% \nocopyrightspace

%%%%%%%%%%%%%%%%%%%%%%%%%%%%%%%%%%%%%%%%%%%%%%%%%%%%%%%%%%%%%%%%
%%%%%%%%%%%%%%%%%%%%%% START OF THE PAPER %%%%%%%%%%%%%%%%%%%%%%
%%%%%%%%%%%%%%%%%%%%%%%%%%%%%%%%%%%%%%%%%%%%%%%%%%%%%%%%%%%%%%%%

\begin{document}

%% The ``\maketitle'' command must be the first command after the
%% ``begin{document}'' command. It prepares and prints the title block.

%% the only exception to this rule is the \firstsection command
\firstsection{Introduction}

\maketitle
%%
Epidemiology aims to characterize health and disease conditions in defined populations (\emph{Cohorts}).
%%
Insights about risk factors allow to characterize disease-specific high-risk groups and act as important diagnostic key figures \cite{Fletcher2012}.
%%
They can also be used to give recommendations regarding a healthy lifestyle and provide information about wide spread diseases.
%%
In the standard workflow, physicians translate observations into hypotheses, which are depicted using epidemiological features and then assessed using regression analyses.

An important epidemiological tool for deriving such features are \emph{Cohort studies}, such as the Study of Health in Pomerania (SHIP) \cite{Volzke2011}.
%%
To reduce any selection bias, subjects are invited at random and without a focus on a specific disease.
%%
The acquired features range from social and lifestyle factors to prior or current diseases and medications as well as medical parameters, such as blood pressure and also comprises of non-radiating medical image data. e.g. magnetic resonance imaging (MRI).
%%
Medical image data quantified using user-defined landmarks, describing for example shape, volume or diameter of a structure.
%%

To assess the statistical resilience of a hypotheses using regression analyses rarely involves more than three features due to the required subject count.
%%
Due to missing overview techniques, possibly interesting correlations lie within the data, but are not made apparent.
%%
Explorative analyses and first overview visualizations of the data set as presented by Klemm et al. \cite{Klemm2014VIS} are not custom tailored to a specific target variable and mostly highlights correlations between variables, which are known to the domain expert (e.g. correlation between body size and spine shape).
%%
We incorporate the regression analysis, which is familiar to the domain experts into a overview visualizations, which can either be used for an hypothesis-free analysis or a analysis towards a specific disease or hypotheses.
%%
This is achieved by providing template regression formulas, which are applied to all potential variable combinations.
%%
Since the notation is familiar to epidemiologists, they can rapidly include their domain knowledge into the analysis process.
%%
Difference views between regression formulas allow to assess the influences of individual variables in the process.

Our contributions are:
\begin{itemize}
	\item A overview visualization technique describing feature interactions using target features.
	\item Visualization techniques, which incorporate overview visualizations of all regression analyses at once as well as details on demand techniques for detailed investigations of feature relationships.
	\item freely adjustable regression formulas provide a simple, yet powerful way to adjust the regression analysis to specific hypotheses about the data.
	\item analysis of confounding variables by providing comparison views between different formula results
	\item The open and web based approach of the system allows for analysis of any data using the presented method.
\end{itemize}

\section{Epidemiological Background} \label{MedicalAndTechnicalBackground}
%%
This section covers the epidemiological workflow and requirements.
%%
\subsection{Epidemiological Workflow} \label{EpidemiologicalWorkflow}
Epidemiology unites experts from different academic disciplines, such as physicians, statisticians and medical computer scientists focusing on biometrics and image segmentation.
%%
Their goal is derive disease-specific risk factors by assessing epidemiological features using statistical methods.
%%
As described by \cite{Thew2009} is divided into three different steps:
\begin{itemize}
	\item Clinicians make observations in the daily practice which are translated into hypotheses.
	\item Epidemiologists compile a list of variables depicting the hypothesis and include confounding variables.
	\item Statisticians assess the association of the derived features towards the investigated disease.
\end{itemize}
%%
Relative risks can be determined if a statistical resilient association of features towards a condition are extracted.
%%
They indicate the per-subject chance of developing the disease.
%%
Reproducibility of results is a epidemiological key requirement and guides all analyses steps.
%%
Statistical programs, such as \texttt{SPSS} are used to analyzed the data using the classical sequential epidemiological workflow, which uses images largely only to communicate results, rather than providing insight.
%%
A alternative data driven approach is described as follows.

\subsubsection{Data Driven Hypothesis-Generation-based Analysis}
Klemm et al. describe an approach for a Interactive Visual Analysis approach for image-centric cohort study data, which connects to the variable listing step \cite{Klemm2014VIS}.
%%
Their methods aim to derive hypotheses through analysis of the data and observations about previously unknown feature correlations.
%%
Employing \emph{hypotheses generation} requires overview visualizations of feature correlations, which are not supported by standard statistical processors.

\subsection{Epidemiological Data} \label{sec:EpidemiologicalData}
%%
\emph{Cohort studies} are a major epidemiological tool for gathering epidemiological features.
%%
They yield a highly heterogenous and incomplete information space.
%%
Subject data are collected with the widest range possible, allowing the data set to be assessed towards different diseases.
%%
The feature space comprises information about lifestyle, somatometric variables, medical parameters, genetic data as well as medical image data derived through different modalities, such as questionnaires or medical examinations or laboratory analyses.
%%
Many features are sparse, such as follow up questions about a medication or treatment of a certain disease.
%%
Other features are exclusive, such as women-specific questions, e.g. number of born children or period status.

\com{Restriction of amount of data because it must not be triangulated and because the ethics committees are very restrictive}

\paragraph{Data Types.}
%%
Medical status variables or lifestyle factors are often of \emph{dichotomous} (binary) type.
%%
The data space comprises continuous variables (somatometric variables, such as body weight or BMI or laboratory values) as well as categorical variables (e.g. graduation type).
%%
The data heterogeneity has to be taken into account when the analysis method is chosen.
%%
Continuous data are often discretized (e.g. 10 year steps for age) to equalize the variable types and to simplify the method selection.
%%
This is however avoided if possible, since it reduces the information space and also introduces a new information bias, as assumptions are modeled through the discretization.
%%

\paragraph{Image Data.}
%%
Since the Rotterdam study, many modern cohort studies include medical image data.
%%
For ethical reasons, the imaging modalities must not include ionizing radiation.
%%
The image quality is often inferior to clinical standard, which is a tradeoff between time and cost \cite{Preim2014}.
%%
These data are hard to analyze as they require segmentation highlighting the structures of interest.
%%
This process is prone to inter and intra-observer variability when carried out manually.
%%
Automatic or semi-automatic solutions bypass this problem, but are costly and need to be custom tailored to the structure of interest \cite{Toennies2015}.
%%
Image-derived variables are either dichotomous variables indicating a medical finding by a radiologist or continuous, describing the shape of segmented structures (e.g. volume or diameter).

\paragraph{Confounding Features}
Features influencing the exposure as well as the outcome of a analysis are called confounders.
%%
The analysis model has to be adjusted by normalizing all included features towards the confounding feature.
%%
\emph{Age} is included as confounder in almost any epidemiological analysis, since most diseases (such as different cancer types) are more likely to happen with increasing age.
%%
It also influences the general body condition and therefore almost all features acquired through cohort studies.
%%
Another important confounder is \emph{Gender}.
%%
Other confounder have to be selected by epidemiologists specific to the investigated condition.

\subsection{Regression Analysis} \label{sec:RegressionAnalysis}
%%
Regression analyses are the most important statistical tool when analyzing epidemiological data.
%%
They are also the foundation of this work.
%%
A regression analysis assesses the influence of one or more (\emph{independent}) features to one target (\emph{dependent}) feature.
%%
The regression model yields a function describing the target feature by weighting the independent features.
%%
Metrics, such as the weightings itself and associated p values and describe the resulting function (the \emph{model}).
%%
R$^2$ values describe the goodness of fit; in other words how well the dependent features describe the target feature.
%%
The value is in the range [0,1], while 1 encodes a perfect fit.

\paragraph{Regression Analysis Notation.} Regression formulas are usually denoted as follows:
%%
\begin{equation}
Dependent \sim Independent_1 + Independent_2 + ... + Independent_n
\label{eq:RegressionNotation}
\end{equation}
%%
The most used regression operators comprise:
%%
\begin{itemize}
	\item $+$ / $-$ inclusion/exclusion of the variable,
	\item $:$ inclusion of interactions between the variables (e.g. $x:y$),
	\item $*$ inclusion the variables as well as their interactions (e.g. $x*y$)
	\item $|$ (conditioning) inclusion of variable x, given y (e.g. $x|y$)
\end{itemize}
%%
The type of the target feature restricts the regression type.

\paragraph{Linear Regression for Continuous Target.} The basic type is the linear regression, creating linear weightings for the \emph{independent} features.
%%
The dependent variable has to be of a continuous type.
%%

\paragraph{Logistic Regression for Dichotomous Target.} Logistic regression implies a dichotomous target variable.
%%
The target is described by fitting a logistic function.
%%
Logistic models do as opposed to linear models does not allow for extracting a R$^2$ goodness of fit value.
%%
Therefore, pseudo-R$^2$ values are extracted, such as the \emph{Nagelkerke R$^2$}, which mimics the behavior of the R$^2$.
%%
\emph{Nagelkerke R$^2$} cannot be compared to R$^2$ values extracted from linear regression model.

%\com{Regression cubes are an important tool for checking statistical resilience.}
%\com{Describe how regression cubes work (including notation).}
\com{Move this to Regression Cube part?}

\subsection{The Study of Health in Pomerania (SHIP)}
The \texttt{SHIP}, located in Northern Germany aims to aims to characterize health and disease in the widest range possible \cite{Volzke2011}.
%%
It does not focus on a specific disease, making the data set open for many diseases.
%%
Unique for the \texttt{SHIP} is the acquisition of medical image data per subject.
%%
A second cohort, \texttt{SHIP-TREND} was introduced in 2012.
%%
Data for both cohorts are examined in a in a 5-year time span.
%%
New parameters are added in each iteration, extending the range of investigated diseases.
%%
For the last acquisition (\texttt{SHIP-2} and \texttt{SHIP-TREND-0}), MRI scans are included into the cohort \cite{Hegenscheid2009, Ivanovska2014}.
%%
%Starting in 1997 with a cohort of 4,308 subjects, the \texttt{SHIP}, located in Northern Germany, aims to characterize health and disease in the widest range possible \cite{Volzke2011}.
%After the pioneering Rotterdam study (started in 1990), several MR imaging study initiatives were initiated.
%
%They slightly differ in clinical focus, acquired data and epidemiological research questions.
%
%Starting in 1997 with a cohort of 4,308 subjects, the \texttt{SHIP}, located in Northern Germany, aims to characterize health and disease in the widest range possible \cite{Volzke2011}.
%
%Data are collected without focus on a group of diseases.
%
%This allows to query the data regarding many diseases and conditions.
%
%
\section{Prior and Related Work}
\com{From VAST'14 Paper:}
\paragraph{Visual Analysis of Heterogenous Data.}
Zhang et al. \cite{Zhang2012} provide a web-based system for analyzing subject groups with linked views and batch-processing capabilities for categorizing new subject entries into the data set.
%
Their definition of a cohort differs from the understanding of the term in an epidemiological context by denoting every parameter-divided subject group as individual cohort.
%
Due to the short paper length, detail is missing on the data types and their algorithms of identifying similar subjects or whether they employ statistical measures.
%
We employ the idea of adding variables via drag and drop into a canvas area.

Generalized Pairs Plots (\texttt{GPLOM´S}) are an information visualization technique comparing heterogenous variables pairwise using a plot-matrix grouped by type \cite{GPLOMS, Francois2013}.
%%
They are useful to gain an overview over numerous variables and their distributions.
%%
Histograms, bar charts, scatter plots and heat maps are used to visualize variable combinations with regard to their type.
%%
The resulting matrix provides an \emph{overview visualization}, but requires a lot of screen space for many variables (127 in our application scenario).
%%
We incorporate the idea of adaptive type-dependent visualizations.
%%
Dai et al. \cite{Dai2005} explored risk factors by incorporating choropleth maps of epidemiological variables (e.g., mortality rates in a region) with parallel coordinates, bar charts and scatter plots with integrated regression lines.
%%
Their findings yielded a \emph{Concept Map}, which linked cancer-related associations via graph edges.
%%
While their goal to identify possible risk factors using socio-economic and health data is similar to ours, they focus on iteratively refining defined hypotheses and on geographical data.
%%
We employ the use of small multiples for incorporating heterogenous data types for comparability.
%%
Chui et al. \cite{Chui2011} visualized associations in time-dependent epidemiological data using time-series plots highlighting risk factor differences in age and gender.
%%
%The connection of time series plots with histograms and heat maps allows for fast insight into complex relationships and 
While the work shows how different visualization techniques provide insight into these data sets, it focuses on the time aspect, which is not present for our data.
%%
% *********************** Our own Work ***********************
% Prior Work
\paragraph{Prior Work.}
We visualized lumbar spine variabilities based on a semi-automatic shape detection algorithm of 490 participants of the \texttt{SHIP-2} cohort \cite{Klemm2013VMV}.
%
Hierarchical agglomerative clustering divided the population into shape-related groups.
%
As proof of concept, a relation between the size of the segmented shape and the measured size of the subjects was shown.
%
This work focuses on incorporating these derived data as new variables, enabling to include it into the hypothesis validation and generation process.
%
When applying clustering techniques to the non-image data it was found that \texttt{k-Prototypes} and \texttt{DBSCAN} are appropriate, but are strongly dependent on the chosen variables and distance measures \cite{Klemm2014BVM}.
%
Niemann et al. \cite{Niemann2014} presented an interactive data mining tool for the assessment of risk factors of hepatic steatosis, the fatty liver disease.
%
Association rules created by data mining methods can be analyzed interactively with their tool and highlight potentially overlooked variables.
% *********************** / Our own Work ***********************
\section{Regression Cube Analysis of Cohort Study Data}
%%
The basic idea of our \emph{Regression Cube} is to provide a overview visualization of large cohort study data sets towards target variables.
%% TODO BLIND
Overview visualizations of feature relationships as presented by Klemm et al. \cite{Klemm2014VIS} are often focused on relationships between the visualized features.
%%
Correlation metrics, such as the \emph{Pearson product-moment correlation coefficient} or \emph{Cram\'{e}r's V} contingency values are incorporated to achieve this goal.
%%
In epidemiology, these relationships are also of interest, but rather towards their explanatory power towards the target feature.
%%
These target features often indicates the presence of the investigated disease.
%%
As described in Section \ref{sec:RegressionAnalysis}, regression analyses are the statistical tool of choice for analyzing these relationships.
%%
A regression model is based on expert knowledge, there is no unified rule on how to apply them to a given set of features, so they have to be applied with care.
%%
\subsection{Cube Description using Regression Formula Notation}
%%
Expert knowledge is introduced into regression analyses using the regression formulas.
%%
As described in Section~\ref{sec:RegressionAnalysis}, the formula input influences the type of the chosen regression method as well as the \emph{independent} features describing the target.
%%
%In order to calculate all combinations of two to three independent features, we can apply  \emph{Regression Cube} aims to 
%%

Since we want to use the regression analyses with a overview visualization, we are are interested in all possible combinations of (two or more) independent features describing a target.
%%
We achieve this by introducing dynamic variables $X$, $Y$ and $Z$ into the regression notation.
%%
The method then replaces the dynamic variables with all features in the data set.
%%
In a data set with 100 features, the regression formula
\begin{equation}
Cancer \sim X + Y
\end{equation}
would yield 10.000 regression models, describing all possible combinations of two features in the data describing the target $Cancer$.
%%
The major advantage of this notation is that it comes natural to anyone familiar with regression analysis, because it uses the same notation, all operators can be used as before.
%%
This allows for a fast adaptation and in the epidemiological application domain.
%%
With simple adjustments to the formula, different results can be achieved:
%%
\begin{itemize}
	\item $Z \sim X + Y$ calculates all combinations of two features towards all possible target features.
	\item $Cancer \sim X + Y + BodyWeight$ includes the $BodyWeight$ feature into all regression models as feature.
	\item $Cancer \sim X + Y + Z$ calculates all combinations of three features towards the target.
\end{itemize}
%%
The problem with this brute-force approach lies in its complexity.
%%
The number calculated regression models increases exponentially for each dynamic variable added.
%%
If we assume a data set with 100 features with the formula $Z \sim X + Y$ we calculate 1,000,000 regression models.
%%
When each regression takes about 50~ms calculation, we have to wait roughly 14 hours for the calculation to complete.
%%
Therefore, the computational complexity needs to be reduced.
%%
%\com{Regression formulas are described using standard regression formula notation by introducing variables.}
%%
%\com{Target Group: Physicians/Epidemiologists with basic statistical background to understand regression formulas and Interaction.}

\subsection{Target-Variable-dependent Dimension Reduction}
\com{Uli, Can you please proofread this?}
%%
The vast majority of features in an epidemiological data have no or a very low relation towards the target feature.
%%
Identifying these features und excluding them from the calculation can reduce the number of dimension significantly.
%%
The \emph{Correlation based Feature Selection} (CFS) algorithm is very popular in data mining for achieving this task \cite{CFS}.
%%
It uses information entropy to select the features which have the most explanatory power towards the target feature.
%%
At the same time it tries to reduce the correlation between the independent features.
%%
When for example the \emph{body weight} has a strong explanatory power towards the target it is likely that \emph{BMI} or \emph{waist circumference} behave similar towards it.
%%
They, however, correlate strongly with each other.
%%
The CFS algorithm would then select the feature which has the largest explanatory power and discards the other dimensions.

We apply the CFS algorithm for each target feature in a regression formula with dynamic variables.
%%
The formula $Cancer \sim X + Y$ would yield one initial CFS information space reduction.
%%
For $Z \sim X + Y$ the CFS algorithm is applied to the data every time $Z$ is replaced with another feature.
%%

The number of features calculated by the CFS algorithm is dependent on the information entropy in the data.
%%
In our epidemiological data we observed a usual number of 10 to 20 features.
%%
For features, such as age or gender, which affect most other features (Recall "Confounder" in Section~\ref{sec:EpidemiologicalData}) the number is significantly larger (about half the features in the whole data set).
%%

Now we have a method to derive the interesting regression models in a reasonable time span.
%%
The next section shows ways of abstracting the results to make them visually feasible.

\subsection{Abstracting Regression Results using R$^2$}
\begin{figure}[htb]
 \centering
 \includegraphics[width=3.0in]{figures/cube_sketch}
 \caption{
 (a) Overview visualization using a heatmap for the formula $Z \sim X + Y$, where $Z$ assumes feature $Age$.
 %%
 The R$^2$ values extracted from the regression formulas depict the goodness of fit and are mapped to color saturation.
 %%
 A saturated color shows a strong correlation.
 %%
 (b) Since $Z$ assumes all features $n$ as given by the formula, it yields $n$ heatmap visualizations.
 %%
 These represent the slices in our cube visualization.
 }
  \label{fig:Cube}
\end{figure}
%%
The goal of a overview visualization is to provide a comprehensive view on the data, which is easy to understand.
%% TODO: Cite another paper here!
As described in \cite{Klemm2014VIS}, correlation values scaled between 0 (no correlation) and 1 (perfect correlation) can be encoded using color on a mosaic plot.
%%
Regression models are more complex, having many associated describing metrics.
%%
For the \emph{Regression Cube} analysis we are interested in the goodness of fit of the resulting model, which allows to infer about the predictive quality of the independent features included in the model.
%%
As described in Section~\ref{sec:RegressionAnalysis}, the R$^2$ value is the metric allowing for this kind of assessment.
%%
Since it is scaled between [0,1] it also allows for comparison between regression models.
%%
We can apply the same information mapping by translating the R$^2$ values to color saturation (Fig.~\ref{fig:Cube} (a)).
%%
This describes a 2D regression square for dynamic variables $X$ and $Y$ (e.g. $Age \sim X + Y$).
%%
Introducing $Z$ creates a 3D \emph{Regression Cube}.
%%
The visualization of R$^2$ values derived from different regression cubes (e.g. $Z \sim X + Y$) is misleading, as they can be compared relatively, but not in precise numbers.
%%
Therefore, the R$^2$ results of different regression methods are encoded using different colors (e.g. blue for linear regression and red for logistic regression).
%%
This way, the cube can easily be extended using other regression types.
%%
For cubes having one fixed target feature, such as $Cancer \sim X + Y + Z$ no such encodings is required and the $z$ dimension can be compared directly.

The goal is to create a overview visualization for a data set, but on the other hand we also want to incorporate expert knowledge into the visualization by adapting the underlying formulas.
%%
These two approaches do not exclude each other, they rather underline the difference in purpose of the chosen formula.
%%
Different analysis approaches require different starting points using the \emph{Regression Cube}.

\subsection{Analysis Workflow}
\begin{figure}[htb]
 \centering
 \includegraphics[width=3.0in]{figures/workflow_sketch}
 \caption{
 Workflow of the analysis using \emph{Regression Cubes}.
 %%
 The analysis starts with a set of features given by the data set.
 %%
 Then, the user may specify a formula regarding to specify a hypothesis, or use a predefined formula to start a explorative analysis of the data set.
 %%
 The Regression Cube is then visualized, where the user has either the option to adjust the formula or to derive details on demand on a specific regression.
 %%
 Insights into the data yield either an adjustment of the current formula.
 %%
 By applying a difference view, regression cubes can be compared to each other.
 %%
 The analysis yields insights and hypothesis about feature relations.
 %%
 }
  \label{fig:Workflow}
\end{figure}

\com{Different Formulas work as different steps in the analysis (Fig.~\ref{fig:Workflow})}
 \com{Workflow for different analysis approaches.}
 \com{0. Select the Variables.}
 \com{1. Select the Formulas}
 \com{a) Analysis without hypothesis about the data set (z ~ x+y)}
 \com{b) Analysis without hypothesis with confounder (z ~ age)}
 \com{c) Analysis with hypothesis about the data set (Var ~ x+y+z)}
 \com{d) Use Difference View}
 \com{d) Clarify using confounder (Var ~ x)}
 \com{3. Find interesting pane.}
 \com{4. Get Details on Demand.}
%%
Analyze first, Show Overview, Details on Demand
%%
\section{System Design and Implementation} \label{Interaction- and Visualization Techniques}
%%

\subsection{Design and Visualization Techniques} \label{Structure and Workflow}
\begin{figure}[htb]
 \centering
 \includegraphics[width=2.8in]{figures/placeholder}
 \caption{
 \com{Screenshot of the cube visualization embedded in the framework}
 \com{Also include active tooltip}
 }
  \label{fig:Cube}
\end{figure}

\begin{figure}[htb]
 \centering
 \includegraphics[width=2.8in]{figures/placeholder}
 \caption{
 \com{Screenshots of Cubes from the Evaluation}
 }
  \label{fig:comparison}
\end{figure}
%%
\com{Simplicity of interface to allow for steep learning curve.}
\com{Visualization as skewed cube to reduce visual clutter and remove double entries.}
\com{Cube acts as visual mini map to give a global impression over the data.}
\com{current pane shown using a heat map visualization.}
\com{details on demand for selected regression formula.}
\com{comparison view using reference cubes.}

\paragraph{System Layout.}

\subsection{Implementation} \label{implementation}
\begin{figure}[htb]
 \centering
 \includegraphics[width=2.8in]{figures/placeholder}
 \caption{
 \com{Describe the Bootstrap/Angular/D3 Frontend, Node backend, R/OCPU Backend.}
 }
  \label{fig:Implementation}
\end{figure}
%%
\com{Use VAST'14 as Guideline}
\com{Speed is merely a matter of available server machines due to parallelization process.}
\com{Security enabled by hashing files.}
\com{Bootstrap/Angular/D3 Frontend, Node backend, R/OCPU Backend (Fig.~\ref{fig:Implementation})}

\section{Application} \label{application}
%
\subsection{The Breast Fat Data Set}
\com{Describe how the data was acquired and preprocessed}

\subsubsection{Data Preprocessing} \label{application:Data Preprocessing}
The data processing follows the description in Section~\ref{Data Preprocessing}.
%

\subsection{The Hepatic Steatosis Data Set}
\com{Uli: Describe the data set and also prior work on it.}

\subsubsection{Data Preprocessing} \label{application:Data Preprocessing}
The data processing follows the description in Section~\ref{Data Preprocessing}.
%

\subsection{Participants, Setup and Procedure}
%%
\com{Use of VDAR Technique}
%
\paragraph{Setup.} Due to the large geographical distance, the evaluation was done completely web-based.
%%
\paragraph{Procedure.}
%%
\subsection{Case 1: Hypothesis-free Analysis of the Breast Cancer Data Set}

\subsection{Case 2: Hypothesis-driven Analysis of the Hepatic Steatosis Data Set}
%%
\subsection{Further Feedback and Lessons Learned} \label{Lessons Learned}
%%
\com{Feedback from the evaluation goes here.}
\com{Time-aspect critical, interactive analysis requires for fast response. The method needs to be speeded up.}

\section{Summary and Conclusion}
\paragraph{Future Work.}
\com{Implementing of regression analysis on the graphics card.}
%%
\begin{small}
	\acknowledgments{Omitted due to blind review}
% \acknowledgments{SHIP is part of the Community Medicine Research net of the University of Greifswald, Germany, which is funded by the Federal Ministry of Education and Research (grant no. 03ZIK012), the Ministry of Cultural Affairs as well as the Social Ministry of the Federal State of Mecklenburg-West Pomerania. Whole-body MR imaging was supported by a joint grant from Siemens Healthcare, Erlangen, Germany and the Federal State of Mecklenburg-Vorpommern. The University of Greifswald is a member of the ‘Centre of Knowledge Interchange’ program of the Siemens AG. This work was supported by the DFG Priority Program 1335: Scalable Visual Analytics. We thank Marko Rak and Klaus Toennies for providing the image detection data.}
\end{small}
\clearpage
\newpage
\bibliographystyle{abbrv}
%%use following if all content of bibtex file should be shown
%\nocite{*}
\bibliography{bibliography}
\end{document}
